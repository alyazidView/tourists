% Options for packages loaded elsewhere
\PassOptionsToPackage{unicode}{hyperref}
\PassOptionsToPackage{hyphens}{url}
%
\documentclass[
  ignorenonframetext,
]{beamer}
\usepackage{pgfpages}
\setbeamertemplate{caption}[numbered]
\setbeamertemplate{caption label separator}{: }
\setbeamercolor{caption name}{fg=normal text.fg}
\beamertemplatenavigationsymbolsempty
% Prevent slide breaks in the middle of a paragraph
\widowpenalties 1 10000
\raggedbottom
\setbeamertemplate{part page}{
  \centering
  \begin{beamercolorbox}[sep=16pt,center]{part title}
    \usebeamerfont{part title}\insertpart\par
  \end{beamercolorbox}
}
\setbeamertemplate{section page}{
  \centering
  \begin{beamercolorbox}[sep=12pt,center]{part title}
    \usebeamerfont{section title}\insertsection\par
  \end{beamercolorbox}
}
\setbeamertemplate{subsection page}{
  \centering
  \begin{beamercolorbox}[sep=8pt,center]{part title}
    \usebeamerfont{subsection title}\insertsubsection\par
  \end{beamercolorbox}
}
\AtBeginPart{
  \frame{\partpage}
}
\AtBeginSection{
  \ifbibliography
  \else
    \frame{\sectionpage}
  \fi
}
\AtBeginSubsection{
  \frame{\subsectionpage}
}

\usepackage{amsmath,amssymb}
\usepackage{lmodern}
\usepackage{iftex}
\ifPDFTeX
  \usepackage[T1]{fontenc}
  \usepackage[utf8]{inputenc}
  \usepackage{textcomp} % provide euro and other symbols
\else % if luatex or xetex
  \usepackage{unicode-math}
  \defaultfontfeatures{Scale=MatchLowercase}
  \defaultfontfeatures[\rmfamily]{Ligatures=TeX,Scale=1}
\fi
% Use upquote if available, for straight quotes in verbatim environments
\IfFileExists{upquote.sty}{\usepackage{upquote}}{}
\IfFileExists{microtype.sty}{% use microtype if available
  \usepackage[]{microtype}
  \UseMicrotypeSet[protrusion]{basicmath} % disable protrusion for tt fonts
}{}
\makeatletter
\@ifundefined{KOMAClassName}{% if non-KOMA class
  \IfFileExists{parskip.sty}{%
    \usepackage{parskip}
  }{% else
    \setlength{\parindent}{0pt}
    \setlength{\parskip}{6pt plus 2pt minus 1pt}}
}{% if KOMA class
  \KOMAoptions{parskip=half}}
\makeatother
\usepackage{xcolor}
\newif\ifbibliography
\setlength{\emergencystretch}{3em} % prevent overfull lines
\setcounter{secnumdepth}{-\maxdimen} % remove section numbering


\providecommand{\tightlist}{%
  \setlength{\itemsep}{0pt}\setlength{\parskip}{0pt}}\usepackage{longtable,booktabs,array}
\usepackage{calc} % for calculating minipage widths
\usepackage{caption}
% Make caption package work with longtable
\makeatletter
\def\fnum@table{\tablename~\thetable}
\makeatother
\usepackage{graphicx}
\makeatletter
\def\maxwidth{\ifdim\Gin@nat@width>\linewidth\linewidth\else\Gin@nat@width\fi}
\def\maxheight{\ifdim\Gin@nat@height>\textheight\textheight\else\Gin@nat@height\fi}
\makeatother
% Scale images if necessary, so that they will not overflow the page
% margins by default, and it is still possible to overwrite the defaults
% using explicit options in \includegraphics[width, height, ...]{}
\setkeys{Gin}{width=\maxwidth,height=\maxheight,keepaspectratio}
% Set default figure placement to htbp
\makeatletter
\def\fps@figure{htbp}
\makeatother

\makeatletter
\makeatother
\makeatletter
\makeatother
\makeatletter
\@ifpackageloaded{caption}{}{\usepackage{caption}}
\AtBeginDocument{%
\ifdefined\contentsname
  \renewcommand*\contentsname{Table of contents}
\else
  \newcommand\contentsname{Table of contents}
\fi
\ifdefined\listfigurename
  \renewcommand*\listfigurename{List of Figures}
\else
  \newcommand\listfigurename{List of Figures}
\fi
\ifdefined\listtablename
  \renewcommand*\listtablename{List of Tables}
\else
  \newcommand\listtablename{List of Tables}
\fi
\ifdefined\figurename
  \renewcommand*\figurename{Figure}
\else
  \newcommand\figurename{Figure}
\fi
\ifdefined\tablename
  \renewcommand*\tablename{Table}
\else
  \newcommand\tablename{Table}
\fi
}
\@ifpackageloaded{float}{}{\usepackage{float}}
\floatstyle{ruled}
\@ifundefined{c@chapter}{\newfloat{codelisting}{h}{lop}}{\newfloat{codelisting}{h}{lop}[chapter]}
\floatname{codelisting}{Listing}
\newcommand*\listoflistings{\listof{codelisting}{List of Listings}}
\makeatother
\makeatletter
\@ifpackageloaded{caption}{}{\usepackage{caption}}
\@ifpackageloaded{subcaption}{}{\usepackage{subcaption}}
\makeatother
\makeatletter
\@ifpackageloaded{tcolorbox}{}{\usepackage[many]{tcolorbox}}
\makeatother
\makeatletter
\@ifundefined{shadecolor}{\definecolor{shadecolor}{rgb}{.97, .97, .97}}
\makeatother
\makeatletter
\makeatother
\ifLuaTeX
  \usepackage{selnolig}  % disable illegal ligatures
\fi
\IfFileExists{bookmark.sty}{\usepackage{bookmark}}{\usepackage{hyperref}}
\IfFileExists{xurl.sty}{\usepackage{xurl}}{} % add URL line breaks if available
\urlstyle{same} % disable monospaced font for URLs
\hypersetup{
  pdftitle={Tourist - Airbnb Canada Quebec},
  pdfauthor={Rehab Alaswad; Rawan Aljohani; Alyazid Alhumaydani; Ibrahim Alghrabi},
  hidelinks,
  pdfcreator={LaTeX via pandoc}}

\title{Tourist - Airbnb Canada Quebec}
\author{Rehab Alaswad \and Rawan Aljohani \and Alyazid
Alhumaydani \and Ibrahim Alghrabi}
\date{}

\begin{document}
\frame{\titlepage}
\ifdefined\Shaded\renewenvironment{Shaded}{\begin{tcolorbox}[frame hidden, interior hidden, enhanced, borderline west={3pt}{0pt}{shadecolor}, sharp corners, boxrule=0pt, breakable]}{\end{tcolorbox}}\fi

\begin{frame}{Introduction:}
\protect\hypertarget{introduction}{}
In this report, we will study the Airbnb Canada dataset and try to make
recommendation of a listing to an investor based on how close the
listing is to nature.

\begin{block}{About the Dataset:}
\protect\hypertarget{about-the-dataset}{}
The dataset was obtained from Inside Airbnb, and it provides data and
advocacy about Airbnb's impact on resdential communities. Inside Airbnb
provides data for many countries but we chose Canda becuase of great
nature in Canada.
\end{block}

\begin{block}{Our Scenario}
\protect\hypertarget{our-scenario}{}
To study the Airbnb hospitality market for an investor interested in
purchasing or building a property in one of the tourist twons near great
nature.
\end{block}

\begin{block}{Data Dictionary:}
\protect\hypertarget{data-dictionary}{}
\end{block}
\end{frame}

\begin{frame}
\begin{longtable}[]{@{}
  >{\raggedright\arraybackslash}p{(\columnwidth - 2\tabcolsep) * \real{0.4762}}
  >{\centering\arraybackslash}p{(\columnwidth - 2\tabcolsep) * \real{0.5238}}@{}}
\toprule()
\begin{minipage}[b]{\linewidth}\raggedright
\emph{Variable}
\end{minipage} & \begin{minipage}[b]{\linewidth}\centering
\emph{Description}
\end{minipage} \\
\midrule()
\endhead
id & Airbnb's unique identifier for the listing. \\
name & Name of the listing. \\
host\_id & Airbnb's unique identifier for the host. \\
host\_name & Name of the host. \\
neighbourhood & Name of the neighbourhood. \\
latitude & Uses the World Geodetic System (WGS84) \\
longitude & Uses the World Geodetic System (WGS84) \\
room\_type & Type of the listing; entire house, private room, or shared
room. \\
price & Price of the listing in Canadian dollars. \\
minimum\_nights & Number of minimum stays in the listing. \\
last\_review & The date of the last review. \\
review\_per\_month & Number of review the listing has over the lifetime
of the listing \\
availability\_365 & The availability of the listing in the next 365
days. \\
number\_of\_review\_itm & Number of review the listing has in the past
12 month. \\
license & The license/permit/registration number. \\
\bottomrule()
\end{longtable}
\end{frame}

\begin{frame}[fragile]{Data Cleaning:}
\protect\hypertarget{data-cleaning}{}
First importing the data and show small sample of the dataset.

\begin{tabular}{lrlrlrlrrlrrrlrrrrl}
\toprule
{} &                  id &                                              name &    host\_id &          host\_name &  neighbourhood\_group &                  neighbourhood &   latitude &  longitude &        room\_type &  price &  minimum\_nights &  number\_of\_reviews & last\_review &  reviews\_per\_month &  calculated\_host\_listings\_count &  availability\_365 &  number\_of\_reviews\_ltm & license \\
\midrule
448   &             4137045 &                       Le Marché Sainte Madeleine! &   21461906 &              Kevin &                  NaN &                   Le Sud-Ouest &  45.480030 & -73.557200 &  Entire home/apt &     80 &              31 &                282 &  2021-04-17 &               2.92 &                               1 &               330 &                      0 &     NaN \\
4329  &            25836089 &  Charming, 4 rooms, Condo on Ste-Catherine Street &  172136210 &               Izzo &                  NaN &  Mercier-Hochelaga-Maisonneuve &  45.556320 & -73.529410 &  Entire home/apt &    457 &               2 &                 38 &  2022-08-28 &               0.79 &                               1 &               363 &                     14 &     NaN \\
12791 &  693528462913359071 &              Welcome to ''Le Sand \& Stone Hotel'' &  474405753 &  Le Sand And Stone &                  NaN &                    Ville-Marie &  45.509439 & -73.564237 &     Private room &    130 &               1 &                  4 &  2022-09-05 &               4.00 &                              28 &               265 &                      4 &     NaN \\
2150  &            16984430 &                   Room 301 | Maison Saint-Vincent &  113699303 &       Maisons \& Co &                  NaN &                    Ville-Marie &  45.506840 & -73.553060 &       Hotel room &    186 &               2 &                 45 &  2022-04-03 &               0.69 &                              25 &               212 &                      1 &     NaN \\
\bottomrule
\end{tabular}

~

\begin{verbatim}
<class 'pandas.core.frame.DataFrame'>
RangeIndex: 13621 entries, 0 to 13620
Data columns (total 18 columns):
 #   Column                          Non-Null Count  Dtype  
---  ------                          --------------  -----  
 0   id                              13621 non-null  int64  
 1   name                            13617 non-null  object 
 2   host_id                         13621 non-null  int64  
 3   host_name                       13620 non-null  object 
 4   neighbourhood_group             0 non-null      float64
 5   neighbourhood                   13621 non-null  object 
 6   latitude                        13621 non-null  float64
 7   longitude                       13621 non-null  float64
 8   room_type                       13621 non-null  object 
 9   price                           13621 non-null  int64  
 10  minimum_nights                  13621 non-null  int64  
 11  number_of_reviews               13621 non-null  int64  
 12  last_review                     11047 non-null  object 
 13  reviews_per_month               11047 non-null  float64
 14  calculated_host_listings_count  13621 non-null  int64  
 15  availability_365                13621 non-null  int64  
 16  number_of_reviews_ltm           13621 non-null  int64  
 17  license                         881 non-null    object 
dtypes: float64(4), int64(8), object(6)
memory usage: 1.9+ MB
\end{verbatim}

This datasets has a total of 13621 observation (rows), and 18 variable
(columns). We note that the neighbourhood\_group has it all entries as
NaN, license column also has many missing values. Since both of these
columns are not of interest we will remove them.

Lets have a look at the data type of each columns.

\begin{tabular}{ll}
\toprule
{} &        0 \\
\midrule
id                             &    int64 \\
name                           &   object \\
host\_id                        &    int64 \\
host\_name                      &   object \\
neighbourhood                  &   object \\
latitude                       &  float64 \\
longitude                      &  float64 \\
room\_type                      &   object \\
price                          &    int64 \\
minimum\_nights                 &    int64 \\
number\_of\_reviews              &    int64 \\
last\_review                    &   object \\
reviews\_per\_month              &  float64 \\
calculated\_host\_listings\_count &    int64 \\
availability\_365               &    int64 \\
number\_of\_reviews\_ltm          &    int64 \\
\bottomrule
\end{tabular}

Since IDs usually are not used in mathematical operation, it would be
wise to change them from integer variable into strings.

\begin{verbatim}
id column type: object
id_host column type: object
\end{verbatim}

Lets look at descriptive statistic of the numerical columns:

\begin{tabular}{lrrrrrrrrr}
\toprule
{} &      latitude &     longitude &          price &  minimum\_nights &  number\_of\_reviews &  reviews\_per\_month &  calculated\_host\_listings\_count &  availability\_365 &  number\_of\_reviews\_ltm \\
\midrule
count &  13621.000000 &  13621.000000 &   13621.000000 &    13621.000000 &       13621.000000 &       11047.000000 &                    13621.000000 &      13621.000000 &           13621.000000 \\
mean  &     45.514491 &    -73.591639 &     169.029073 &       15.623523 &          23.994200 &           1.498029 &                       10.141473 &        128.566405 &               8.402760 \\
std   &      0.032831 &      0.051336 &     958.044866 &      108.486987 &          49.724181 &           1.873254 &                       24.852071 &        133.881257 &              16.396578 \\
min   &     45.382600 &    -73.974520 &       0.000000 &        1.000000 &           0.000000 &           0.010000 &                        1.000000 &          0.000000 &               0.000000 \\
25\%   &     45.495850 &    -73.604150 &      65.000000 &        1.000000 &           1.000000 &           0.160000 &                        1.000000 &          0.000000 &               0.000000 \\
50\%   &     45.516270 &    -73.578870 &     109.000000 &        2.000000 &           6.000000 &           0.730000 &                        2.000000 &         79.000000 &               1.000000 \\
75\%   &     45.532630 &    -73.564110 &     176.000000 &       21.000000 &          24.000000 &           2.210000 &                        6.000000 &        258.000000 &               9.000000 \\
max   &     45.713800 &    -73.481630 &  105634.000000 &    11684.000000 &         734.000000 &          24.730000 &                      183.000000 &        365.000000 &             251.000000 \\
\bottomrule
\end{tabular}

~

From the above table, we note the following:

\begin{itemize}
\tightlist
\item
  The maximum number of minimum nights 11684, where the mean is around
  15 nights.
\item
  The maximum price is 105634 \$ with mean of 169 \$ and standard
  variation of 958 \$.
\end{itemize}

Lets investigate these extream points

\begin{verbatim}
The number of observation with price greater than 4000$ is: 15
\end{verbatim}

Both of these listing are private rooms, it seems those entries are an
error so we will remove them. Also in this study I will only include
listing of price less than 4000 \$, since they only represent 15
observation out of 13600.

Distinct neighbourhood in the datasets.

\begin{verbatim}
Number of distinct neighbourhood: 33
\end{verbatim}

\begin{verbatim}
array(['Le Plateau-Mont-Royal', 'Le Sud-Ouest',
       'Villeray-Saint-Michel-Parc-Extension', 'Ville-Marie',
       'Rosemont-La Petite-Patrie', 'LaSalle',
       'Mercier-Hochelaga-Maisonneuve',
       'Côte-des-Neiges-Notre-Dame-de-Grâce', 'Montréal-Ouest', 'Verdun',
       "Baie-d'Urfé", 'Lachine', 'Côte-Saint-Luc',
       'Ahuntsic-Cartierville', 'Saint-Laurent', 'Outremont',
       'Pierrefonds-Roxboro', 'Saint-Léonard', 'Anjou',
       'Rivière-des-Prairies-Pointe-aux-Trembles', 'Pointe-Claire',
       'Mont-Royal', 'Dollard-des-Ormeaux', 'Hampstead', 'Dorval',
       'Westmount', 'Kirkland', "L'Île-Bizard-Sainte-Geneviève",
       'Montréal-Nord', 'Beaconsfield', 'Sainte-Anne-de-Bellevue',
       'Montréal-Est', 'Senneville'], dtype=object)
\end{verbatim}
\end{frame}

\begin{frame}{Supporting Dataset:}
\protect\hypertarget{supporting-dataset}{}
This dataset presents the place of interest in the City of Montreal. The
only places of interest to us are parks, so we can make our
recommendation based on listing close to nature. We will drop any
observation of missing values.

\begin{tabular}{lrlllllllrlllllrr}
\toprule
{} &    ID &              Famille &                  Catégorie &                                       Nom français &                                          Nom court &                  Type & Numéro &                                 rue &  Étage & Bureau &     Ville & Code postal &                 Arrondissement & Classification &  Longitude &   Latitude \\
\midrule
528  &   566 &  Récréatif / sportif &  Parc et autre espace vert &        Parc et site archéologique des Saints-Anges &                              Parc des Saints-Anges &                  Parc &        &                                     &      0 &        &  Montréal &     H8R 3Z7 &                        LaSalle &       niveau 2 & -73.656784 &  45.424198 \\
1085 &  1148 &  Récréatif / sportif &  Parc et autre espace vert &  Jardin communautaire de l’Institut universitai... &  Jardin communautaire de l’Institut universitai... &  Jardin communautaire &        &  Rue de Marseille et Rue du Trianon &      0 &        &  Montréal &             &  Mercier–Hochelaga-Maisonneuve &       niveau 4 & -73.533211 &  45.589813 \\
1654 &  1777 &  Récréatif / sportif &  Parc et autre espace vert &      Jardin communautaire l’Églantier (biologique) &                   Jardin communautaire l’Églantier &  Jardin communautaire &        &    31e Avenue et Boulevard Rosemont &      0 &        &  Montréal &             &      Rosemont–La Petite-Patrie &       niveau 4 & -73.568429 &  45.565778 \\
2650 &  2865 &  Récréatif / sportif &  Parc et autre espace vert &               Plateau de travail (Circuit Jardins) &                                 Plateau de travail &  Jardin communautaire &   1872 &                     Rue Saint-André &      0 &        &  Montréal &             &                    Ville-Marie &       niveau 4 & -73.563608 &  45.518556 \\
\bottomrule
\end{tabular}

There is only four parks in this dataset. We will use the latitude, and
longitude columns of the new dataset, to make two new variable in the
original dataset. The new columns are: `distance\_to\_nearest\_park',
`nearest\_park'

\begin{itemize}
\tightlist
\item
  nearest\_park: the nearest park.
\item
  distance\_to\_nearest\_park: distance of the closet park in km.
\end{itemize}

The equation used will calculate the orthodromic distance (i.e.~the
shortest distance between two points on earth's surface), meaning it
will not take into acoount the road distance or traffic.

\begin{tabular}{llllllrrlrrrlrrrrrl}
\toprule
{} &        id &                                              name &    host\_id & host\_name &            neighbourhood &  latitude &  longitude &        room\_type &  price &  minimum\_nights &  number\_of\_reviews & last\_review &  reviews\_per\_month &  calculated\_host\_listings\_count &  availability\_365 &  number\_of\_reviews\_ltm &  distance\_to\_nearest\_park & nearest\_park \\
\midrule
3164  &  21218315 &                      Cozy room downtown Montreal! &  143819680 &     Jamie &    Le Plateau-Mont-Royal &  45.50776 &  -73.57682 &     Private room &     32 &               9 &                  1 &  2017-10-09 &               0.02 &                               1 &                 0 &                      0 &                  2.545039 &       Park 4 \\
13460 &  52334907 &  Modern retreat. Immersed in nature close to city &   46402818 &     Scott &  Sainte-Anne-de-Bellevue &  45.39960 &  -73.96234 &  Entire home/apt &    265 &               7 &                  5 &  2022-04-01 &               0.58 &                               1 &               149 &                      5 &                 38.636841 &       Park 1 \\
\bottomrule
\end{tabular}
\end{frame}

\begin{frame}[fragile]{EDA:}
\protect\hypertarget{eda}{}
\begin{block}{Price Distribution:}
\protect\hypertarget{price-distribution}{}
We would like the distribution of price, where the distribution is
located and how much is the spread.

\begin{verbatim}
Unable to display output for mime type(s): text/html
\end{verbatim}

\begin{verbatim}
Unable to display output for mime type(s): text/html
\end{verbatim}

\hfill\break

We note the price distribution have high variation. The distribution is
postively skewed, so the mean price may not be a godd representative of
the data.
\end{block}
\end{frame}

\begin{frame}[fragile]{Minimum Nights Distributions:}
\protect\hypertarget{minimum-nights-distributions}{}
Let's have a look at the distribution of minimum nights of the listings,
showing only up to minimum of 180 nights:

\begin{verbatim}
Unable to display output for mime type(s): text/html
\end{verbatim}

~ We can see that, most listing rent from 1 \textasciitilde{} 3 nights,
then there are spike in listings with integer multiple of 30 days
(i.e.~30, 60, 90 days).
\end{frame}

\begin{frame}[fragile]{Listing and Parks Locations on Map:}
\protect\hypertarget{listing-and-parks-locations-on-map}{}
We would like how the listing is distributed on the map, and where the
location of the parks in the area.

\begin{verbatim}
<folium.folium.Map at 0x13761cb20>
\end{verbatim}

~

The heatmap represent the concentration of listing, where the green icon
represent the parks. We can see the park number 4 is overlapping with
high concetration of listing. After seeing what it's look on the map, we
would like to quantify the results, and make a recommendations.
\end{frame}

\begin{frame}[fragile]{Nearest Parks and Distance to Nearest Parks:}
\protect\hypertarget{nearest-parks-and-distance-to-nearest-parks}{}
Lets have a look at the new columns descriptive statistic:

\begin{verbatim}
count    13605.000000
mean         5.463882
std          4.786292
min          0.019696
25%          2.480004
50%          4.250960
75%          7.183879
max         41.689533
Name: distance_to_nearest_park, dtype: float64
\end{verbatim}

\begin{verbatim}
Unable to display output for mime type(s): text/html
\end{verbatim}

~ Note most llisting have nearest park Note that the distance between
most listing and their respective nearest park are around 5.5 km. Also,
the range of distance to nearest park is from 20 m up to 41.7 km.

Lets see what park is close to most listing

\begin{verbatim}
Unable to display output for mime type(s): text/html
\end{verbatim}

\hfill\break

We see that park number 4 has the highest nearest listings, then park 3
with significt drop from 10000 to 2000 listings, then park 2, and
finally park number 1.

~

Afterward, we would like to look the nearest neighbourhood to any park.
We will use the mean of distance to nearest park for each neighbourhood,
and show the six lowest distance mean.

\begin{verbatim}
Unable to display output for mime type(s): text/html
\end{verbatim}

~ The color in the above grpah represent the number of listing in the
neighbourhood. As shown, Le Plateau-Mont-Royal has the lowest mean of
all neighbourhood.
\end{frame}

\begin{frame}{Conclusion}
\protect\hypertarget{conclusion}{}
In conclusion, we use two datasets acquired from both the Inside Airbnb,
and canada goverment website. We were able to analyzed the data
available. Since the investor want the listing close to parks, we would
recommend investing on listing from the following neighbourhoods:

\begin{enumerate}
[1)]
\tightlist
\item
  Le Plateau-Mont-Royal
\item
  Ville-Marie
\item
  Mercier-Hochelaga-Maisonneuve
\end{enumerate}
\end{frame}

\begin{frame}{Sources}
\protect\hypertarget{sources}{}
\begin{itemize}
\tightlist
\item
  \href{http://insideairbnb.com/explore}{Explore}~or~\href{http://insideairbnb.com/get-the-data}{Get}~the
  data
\item
  Read~\href{http://insideairbnb.com/data-policies}{Data Policies}
\item
  Make a~\href{http://insideairbnb.com/data-requests}{Data Request}
\item
  Read~\href{http://insideairbnb.com/data-assumptions}{Data Assumptions}
\item
  View
  the~\href{https://docs.google.com/spreadsheets/d/1iWCNJcSutYqpULSQHlNyGInUvHg2BoUGoNRIGa6Szc4/edit?usp=sharing}{Data
  Dictionary}
\item
  Join the~\href{http://insideairbnb.com/data-community}{Data Community}
\item
  Supplying
  \href{https://open.canada.ca/data/en/dataset/763fe3b8-cdc3-4b8a-bbbd-a0a9bc587c56}{dataset}
\item
  \href{https://www.geeksforgeeks.org/program-distance-two-points-earth/}{Distance
  calculation code}
\end{itemize}
\end{frame}



\end{document}
